\chapter{Introduction}
\section{Motivation}
Emergencies can happen at any time. From minor issues to severe trauma, It is essential that the response teams that come to help them are equipped with the necessary technology to save their lives. There is an increasing demand for Emergency medical services (EMS). To respond to this growing demand, the EMS communities require adaptable and sustainable systems. First Responders or paramedics, as they are generally called in United States, provide rapid assessments and treatment of the sick and injured prior to transporting them to a hospital for further evaluation. Considerable knowledge, skill, and judgment are required to provide quality ambulatory emergency medical services. High quality emergency medical services and paramedics are important part of any health care system.Emergency patients with life-threatening trauma or disease are treated by paramedics  at the scene and during transport. Paramedics often are first to arrive at the scene, and  are allowed to defibrillate, to intubate endotracheally, and to administer life-saving drugs (epinephrine endotracheally, glucose intravenously, etc.). They treat patients at the scene and during transport in the ambulance. Many studies of pre-hospital services place greater emphasis on human factors, efficiency and continuous refinement of standards of practice.EMS providers are expected to render effective life-saving evaluation, intervention, stabilization and treatment that are consistent with existing standards and codes. Paramedics must be properly trained and equipped with suitable equipment, tools and communication systems to ensure that the highest quality of care, safety and reliability are attained before, during and after a medical emergency event (see the references [8-17]).

These situation require responders to effectively care for patients with limited resources and medical infrastructure within the limited space inside the ambulance, often under intense time pressure. For years, emergency medical service providers conducted patient care by manually measuring vital signs, documenting assessments on paper, and communicating over handheld radios. The ability to automate these tasks could greatly relieve the workload for each responder, increase the quality and quantity of patient care, and more efficiently deliver patients to the hospital. 
Steady advances in wireless networking, medical sensors, and interoperability software create exciting possibilities for improving the way we provide emergency care. BlueBox is a project that explores and showcases how these advances in technology can be employed to assist patients and responders in times of emergency. 

\section{Related Work}

\section{Objective}
When emergency occurs, the chaotic setting of limited resources, unreliable communication infrastructure, and inadequate information produces an organizational nightmare for care provider teams and prevents them from providing quality trauma care [1][2]. 
During these health emergencies, when time is of the essence, there is little tolerance for system errors and poor usability designs. 
Collaboration with LABIOMED prompted us to develop BlueBox (like Black Box device in a Aeroplanes) a low-cost, low-power wearable device that automates monitoring patient’s vitals throughout each step of the response process. It provides a new patient care paradigm to the emergency response arena through automation of the patient monitoring and tracking process. The idea is to design an easy to use wearable hardware and integrated software solutions that records important  patient’s vitals and paramedic's logs in the database , which allows the doctors and physicians at the hospital to review the patient information . Access to this information could allow them to provide accurate care.

BlueBox facilitates collaborative and time-critical patient care in the emergency response community. BlueBox  platform support recording of variety of data including Electrocardiogram(ECG), body temperature, Chest movements (using Inertial Measurement Unit sensors) , Environment temperature sensors, paramedic’s voice log about patient condition . BlueBox automates the tasks of data recording and  greatly relieve the workload for each responder, which could improve the quality and quantity of patient care, and more efficiently deliver patients to the hospital with a clear picture of patients condition from the time emergency. Our system accomplishes this through the following technologies:
-	Wearable sensors to sense and record vital signs into an electronic storage. This dramatically improves the current time-consuming process of manually recording vital signs onto paper pre-hospital care reports and then converting the reports into electronic form for the hospitals.
-	Ability to record the paramedic’s comments about the condition of the patient and synchronize with the vitals.

Collaboration with LA BioMed , physicians , paramedics and doctors helped us understand the user needs.

//TODO add references to bibiliography