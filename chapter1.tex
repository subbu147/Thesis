\chapter{Introduction}
\section{Motivation}


Emergencies can happen at any time. From minor issues to severe
trauma, it is essential that the response teams that come to help
them are equipped with the necessary technology to save lives. There
is an increasing demand for Emergency medical services (EMS). To
respond to this growing demand, the EMS communities require adaptable
and sustainable systems. First Responders or paramedics, as they are
generally called in United States, provide rapid assessments and
treatment of the sick while transporting them to a hospital for
further evaluation. Considerable knowledge, skill, and suitable
technology are required to provide quality ambulatory emergency
medical services. High-quality emergency medical services and
paramedics are important parts of any health care system. Paramedics
often are first to arrive at the scene, and are allowed to
defibrillate, to intubate endotracheally, and to administer
life-saving drugs (epinephrine endotracheally, glucose intravenously,
etc.). They treat patients during transport in the ambulance. Many
studies of pre-hospital services place greater emphasis on technology
factors, human efficiency and continuous refinement of standards of
practice. EMS providers are expected to render effective life-saving
evaluation, intervention, stabilization and treatment that are
consistent with existing standards and codes. Paramedics must be
properly trained and suitably equipped with tools and systems to
ensure that the highest quality of care, safety and reliability are
attained before, during and after a medical emergency event
\cite{EMS, EMS1, EMS2, EMS3}.

These situations require responders to effectively care for patients
with limited resources and medical infrastructure within the limited
space inside the ambulance, often under intense time pressure. For
years, emergency medical service providers conducted patient care by
manually measuring vital signs, documenting assessments on paper, and
communicating over hand-held radios. The ability to automate these
tasks could greatly relieve the workload for each responder, increase
the quality and quantity of patient care, and more efficiently
deliver patients to the hospital.  Steady advances in wireless
networking, medical sensors, and interoperability software create
exciting possibilities for improving the way we provide emergency
care. BlueBox is a project that explores and showcases how these
advances in technology can be employed to assist patients and
paramedics in times of emergency. 

\section{Related Work}

\subsection{Medical Information Tag}

Tia Gao et al., developed MiTag (Medical information tag) \ref{fig:miTag}, a cost-effective 
wireless sensor platform that automatically tracks patients 
throughout each step of the emergency response process, from disaster scenes, to ambulances, to hospitals \cite{miTag}.

\begin{figure}
	\centering
	\includegraphics[scale = 0.6 ]{miTag}
	\caption{Medical Information Tag. \cite{miTag}\label{fig:miTag}}
\end{figure}

The miTags can increase the patient care capacity of responders in
the field. The miTag supports sensor add-ons -- GPS, pulse oximetery,
blood pressure and ECG.  The miTag system is out-of-the-box
operational and includes the following key technologies: 1)
cost-effective sensor hardware, 2) self-organizing wireless network,
and 3) scalable server software that analyzes sensor data and
delivers real-time updates to hand-held devices and web portals. The
drawback of this setup is its bulky size and uncomfortable usability.
The miTag system has wireless blood pressure cuff with an additional
9\,V battery to power the cuff which makes it bulky. The miTag uses
wireless sensors (ECG, GPS, Pulse oximeter), each of which has its
own radio and battery, which makes each individual sensor bulky. This
distributed nature of the system adds lot more work for paramedics to
manage each one of them individually, which sometimes could be a
pain. 


\subsection{Advanced Health
	and Disaster Aid Network (AID-N)}

The Advanced Health and Disaster Aid Network (AID-N), developed by
Tia Gao et al.~\cite{AID-N} at The Johns Hopkins University Applied
Physics Laboratory, explores and showcases how these advances in
technology can be employed to assist victims and responders in times
of emergency. AID-N uses open-standard software and best-of-breed
hardware to deliver a scalable, open, and reliable infrastructure
that paves the way for the development of new capabilities and the
extension of existing technologies.  Wearable sensors designed by the
CodeBlue project at Harvard University is used in AID-N. The AID-N
system consists if a pulse oximeter sensor on patient's fingure that
measures heart rate and blood oxygenation (SpO\textsubscript{2})
level of the patient. It uses Wearable sensors to sense and record
vital signs into an electronic patient record database. The vitals
are transmitted to a local base staion (a tablet) that displays the
vital in real time. Compared to MiTag, this system requires very
little setup as it has a single wearable setup. AID-N is integrated
with pre-hospital patient care framework MICHAELS. Due to the chaotic
nature of emergencies, AID-N system faces the challenge of operating
in situations that challenge instrumentation designed for use in the
controlled environment of a clinical situation. 


\section{Objective}

The miTag and the AID-N systems presented above has challenges when
it comes to the usability under a chaotic environment of emergency.
Time is a critical resource during emergency and usability issue
consumes this critical resource. Also the basic idea of building such
devices is to help paramedics provide emergency care while
maintaining record of the patients condition. But the above mentioned
system does not provide space for paramedics to log about the
patients response to their emergency treatment. BlueBox, the proposed
system, addresses these issues.  We developed BlueBox analogous to
Black Box device on airplanes, to be a low-cost, low-power wearable
device that automates monitoring patient's vitals and paramedic's
voice logs throughout each step of the response process. The idea is
to design an easy-to-use wearable hardware and integrate software
solutions to record the patient's vitals and the paramedic's logs in
the storage, which allows the doctors and physicians at the hospital
to review. Access to this information could allow them to provide
better care.

BlueBox platform supports recording of a variety of data including
electrocardiogram (ECG), body temperature, respiration pattern
(through chest movements monitoring and breathing audio), environment
temperature sensors, and the paramedics' speaking voice logs about
patient condition.  BlueBox relieves the workload for paramedics by
automating the task of data recording that helps improve quality of
patient care and deliver patients to the hospital with a clear
picture of patients condition from the time emergency. 

\section{Thesis Goals}

The goal of this thesis is to build low power firmware to realize the
functionalities of the BlueBox device within the resource
constraints. BlueBox is a wearable, battery-operated, easy-to-use
device. Wearable medical devices have strict size and weight
constraints while having a functional and behavioral specifications
that demand the system to have high performance and function for
hours. Battery capacity is directly proportional to its weight as
lithium polymer battery tend to have a fixed energy density. Smaller
the size and weight, lower the capacity of the battery. Battery
capacity is a huge bottleneck for these systems as the system
requires to do active sensing and storage for hours. So these systems
require an extensive power management strategies to be deployed in
hardware and software levels. This problem can be cast as a
hardware-software co-design problem.

Hence, in this thesis project, the goal can be stated as follows:
to build low-power firmware for BlueBox system so that the device can
acquire data at the required frequency and to store the data in local
storage while functioning for at least 4 hours. 
 
 The methods applied to accomplish the project can be elaborated as the following steps:
 \begin{itemize}
	\item signal requirements and constraints. 
 	
	\item and system design requirements to fulfill the behavioral
		specification and mechanical constraints.
 	
	\item design to optimize the required resource and to implement the
		low-power data acquisition firmware.
 	
	\item interfacing software for data retrieval and data
		visualization on the PC.
 	
	\item conducting experiments with the firmware and the software,
		along with the BlueBox system.
 \end{itemize}
 
\section{Thesis Organization}
The rest of this thesis is organized as follows:
\begin{itemize}
	\item Chapter 2 gives a short overview of the BlueBox system. It also describes the significance and requirements of the signal that needs to be recorded by the BlueBox system. This chapter also presents some design requirement confronted with some constraints
	
	\item Chapter 3 provides the final system overview and presents the hardware architecture. This chapter also talks about the important subsystems, sensors and battery of the BlueBox system.
	
	\item Chapter 4 explains the firmware architecture of the BlueBox system and describes its core functionalities. This chapter also talks about procedure for importing and visualizing the acquired data.
	
	\item Chapter 5 gives a background on general sources of power consumption and covers in details the power management techniques implemented in BlueBox firmware.
	
	\item Chapter 6 presents system setup, testing and verification of the BlueBox system. 
	
	\item
	Chapter 7 completes this report by giving the final conclusion and ideas for future enhancement of the device.
	
\end{itemize}
\nomenclature{EMS}{ Emergency Medical Services}