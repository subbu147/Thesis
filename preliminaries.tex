\thesistitle{Low-Power Firmware Design Techniques for ???}

\degreename{Master of Science}

% Use the wording given in the official list of degrees awarded by UCI:
% http://www.rgs.uci.edu/grad/academic/degrees_offered.htm
\degreefield{Computer Engineering}

% Your name as it appears on official UCI records.
\authorname{Subramanian Meenakshi Sundaram}

% Use the full name of each committee member.
\committeechair{Professor Pai H Chou}
\othercommitteemembers
{
  Assistant Professor Mohammad Al Faruque\\
  Professor Rainer D{\"o}mer
}

\degreeyear{2017}

\copyrightdeclaration
{
  {\copyright} {2017} \Authorname
}

% If you have previously published parts of your manuscript, you must list the
% copyright holders; see Section 3.2 of the UCI Thesis and Dissertation Manual.
% Otherwise, this section may be omitted.
% \prepublishedcopyrightdeclaration
% {
% 	Chapter 4 {\copyright} 2003 Springer-Verlag \\
% 	Portion of Chapter 5 {\copyright} 1999 John Wiley \& Sons, Inc. \\
% 	All other materials {\copyright} {\Degreeyear} \Authorname
% }

% The dedication page is optional.
%\dedications
%{
 % (Optional dedication page)
  
  %To ...
%}

\acknowledgments
{
  I would first like to thank Prof. Pai Chou , my advisor for allowing me to take part in this project and for his continuous support and guidance throughout my Master's Thesis.Without his guidance and persistent help this dissertation would not have been possible. I wish to thank the members of my thesis committee: Professor Mohammad Al Faruque and Professor Rainer D{\"o}mer for generously
  offering their time, support, guidance and good will throughout the
  preparation and review of this thesis. I offer my warmful thanks to Dr.Chang, Ruey-Kang from LA BioMed for his guidance throughout the project accomplishments , and helping us in testing the project.I would alo like to thank Eva Villa, LA BioMed, who helped in conducting the field test of the device. I would like to also thank all of my family members,friends, and colleagues, especially  Hsin-Chung Chen, Ph.D., Jun Luan, Ph.D., and Rohit Zambre, Ph.D.,
  who have helped and supported me during my Master's program at UC Irvine.   
}


% Some custom commands for your list of publications and software.
\newcommand{\mypubentry}[3]{
  \begin{tabular*}{1\textwidth}{@{\extracolsep{\fill}}p{4.5in}r}
    \textbf{#1} & \textbf{#2} \\ 
    \multicolumn{2}{@{\extracolsep{\fill}}p{.95\textwidth}}{#3}\vspace{6pt} \\
  \end{tabular*}
}
\newcommand{\mysoftentry}[3]{
  \begin{tabular*}{1\textwidth}{@{\extracolsep{\fill}}lr}
    \textbf{#1} & \url{#2} \\
    \multicolumn{2}{@{\extracolsep{\fill}}p{.95\textwidth}}
    {\emph{#3}}\vspace{-6pt} \\
  \end{tabular*}
}

% The abstract should not be over 350 words, although that's
% supposedly somewhat of a soft constraint.
\thesisabstract
{
  Embedded wearable medical systems design has been a continuously developing research field. The requirements associated with wearable devices posts severe restrictions on size and battery capacity. Even though battery technology is improving continously, the battery weight and capacity are the issues that influence heavily on how wearable medical devices can be used. Medical devices often require real-time processing capabilities and this demands high throughput. The available energy resource of these system within the posted constraints doesn't directly meet the demanding requirements. It requires rigorous resource and power management from the firmware to make it feasible. In this thesis we discuss the power constraints of our wearable emergency medical care monitoring device , BlueBox, and our approach in designing low-power firmware for making the optimal use the resource towards fulfilling the demanding  requirements.
}


%%% Local Variables: ***
%%% mode: latex ***
%%% TeX-master: "thesis.tex" ***
%%% End: ***
