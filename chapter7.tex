\chapter{Conclusion}
A wearable emergency care deice for ambulance has been motivated, designed and developed. The initial phase of the Clinical testing has been done using mannequin-based patient simulation environment. Further clinical testing of the device is being done currently. In this last Chapter, the ideas for the future work is presented.

\section{Discussion}
While the preliminary results suggests that the BlueBox device could be a good viable option for usage in ambulance for emergency monitoring, there are number of things learnt during the design process to improve the system for the future. First, the respiration audio acquired contained noise and other signals including the environment sound. The microphone used for sensing the respiration audio is sensitive to environment noise, including the paramedic's log. Attempt was made at post processing to remove the noise through complex filtering operations but still a clear respiration audio was not reconstructed. Removing the noise at the source seemed to be a good solution for the problem. Then the respiration microphone was placed inside a stethoscope chestpiece, that amplified the breathe sound and attenuated the environment noise. The revision of microphone solved the problem of breathe signal acquisition, but the chestpiece of a stethoscope is heavy and is not really comfortable to wear. An alternate sensor that serves the exact purpose needs to designed inorder make it comfortable to wear. The other problem was with the noise in the ECG signal that appeared due the harmonics from the source. A simple filter in the post processing filtered the noise and solved the issue. 

There were two primary usability issue with the device. The first issue is that because the device does not transmit to a base station( tablet, mobile device), there is no live feedback about the signal quality and it is too hard to check if the all the connections and placement of electrodes are proper. The board has to be placed on the body and run during the test period. Then the micro SD card from the board is removed and the data can be analyzed on the computer. A live feature with the radio transmission meant for debugging help to confirm the signal quality. And the second is that because the central BlueBox board was made using a rigid PCB it makes the device sometimes uncomfortable to fit it on the patients. A flexible PCB could be a required future work to make the device take different shapes and become comfortable to wear. 

\section{Future Work}
The basic proof of concept functionality of BlueBox has been shown with the mannequin-based patient simulation environment. Further clinical tests on real patients is necessary to further validate the device. Further improvements of the device needs to be done to meet the medical grade requirements.

Future work should investigate on the design of proper sensor to sense the breathing pattern of the patient while making sure that the designed sensor setup is comfortable to wear. Also the future work should also research on real time transmission of data to a base station, while making sure that battery life of the device is maintained as per the requirement. It is also required to evaluate the possibility of implementing the system on a flexible PCB within the constrained size and dimension. Additionally, a properly molded enclosure for the PCB could help improve the patient's acceptability of wearing the device. 
\section{Summary}
Wearable emergency monitor can inform physicians about the condition of the patient while the patient was in transit to the hospital. They provide physicians additional information such as the paramedic's logs on patient's response to emergency medications. In this thesis BlueBox, a wearable emergency care device, was discussed and the low power firmware for this device was developed. This device can be worn by patients inside an ambulance on their way to a hospital. Previous works on such devices did not focus on adding and synchronizing the paramedic's logs unlike BlueBox. The implementation of the device in this thesis is able to record the essential data as described in section \ref{overview} for over 5 hours. Preliminary clinical results suggest that it is a potential system to be used in ambulance for Emergency Health Monitoring.