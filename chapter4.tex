\chapter{Firmware Architecture}\label{firmware_arch}

This chapter describes the firmware architecture of the BlueBox system
and describes its core functionalities. This chapter also talks about
importing and visualizing the acquired data.

\section{Overview}

The firmware for the system is written in C and assembly using Code
Composer Studio IDE. TI provides DSP/BIOS (in later versions called
SYS/BIOS) a real-time operating system for their DSPs. This operating
system provides a wide range of system services to an embedded
application including preemptive multitasking, memory management, and
real-time analysis. The RTOS is feature-rich and helps speed up
development and testing of firmware. However, a potential drawback of
using the RTOS is that it trades performance for structure of the DSP
software for quicker and easier development. For a resource and power
constrained application, it is required to squeeze out the
performance from the system for firmware to have complete control
over the underlying hardware. Thus, firmware for the system is
implemented on bare metal using C and using the chip support library
(CSL) for C5515 provided by TI. 


The functions of firmware can be categorized as follows:
 \begin{itemize}
 	\item Initializing System and Sensors 
 	\item Data collection and storage
 	\item Power management 
 \end{itemize}
\FIG{main} gives the overall flow of the firmware.
 \begin{figure}
	\centering
	\includegraphics[scale = 1 ]{main}
\caption{Main thread}
\label{main}
\end{figure}

\section{Initializing System}

The first thing the firmware does after boot-up is to initialize the
clock by configuring the phase locked loop (PLL) control registers.
PLL is configured to generate a 100~MHz clock to tick the system.
After the clocks have been set up, the multiplexed I/O pins are
configured to function as a specific peripheral or general-purpose
input/output (GPIO) pins. Then, the clock and power to those I/O pins
and peripherals are enabled. After the I/O pins are configured, the
communication peripherals such as SPI, I$^2$C, and I2S are
configured. Their mode of operation (master or slave), frequency of
operation, etc., are initialized and configured. DSP's I2S module is
configured as I2S master (master supplies bit clock and word clock)
and the Audio codec AIC3204 is configured as I2S slave. 

\section{Initializing sensor}

Audio Codec AIC3204 is configured, through the I2C interface, to
sample the audio signal at 12~KHz on each of the two channels and
configured to amplify the signal with automatic gain correction
feature enabled. The codec is also configured to filter the audio
signal using first order IIR filter and 5 Bi-quad filters to improve
the audio quality. 

The ECG analog front-end ADS1292R is configured to sample ECG signal
at 500 samples per second. ADS1292R is configured to generate a
hardware interrupt to inform the DSP when the sample is ready so that
the DSP can read the data from the ECG chip over SPI.
%The ECG data is again stored in a ping-pong buffer and the data is
%stored in the permanent storage with the same mechanism as mentioned
%earlier. 
The environment temperature sensor STS21 is configured to sample the
temperature with an 11-bit precision. This data is acquired by
polling.  For every 250 ECG interrupts the STS21 is polled once to
collect the temperature data.  The Analog-to-Digital converters (ADC)
on board are configured to sample the body temperature once every 250
interrupts of ECG.

\section{Data Collection and storage}

The firmware is designed for concurrent data acquisition using
dedicated hardware resources for the signals of interest. These
hardware resources can perform certain required but very defined
functionalities like data buffering, filtering, and data moving
without intervention by the DSP. These hardware accelerators requires
the DSP to configure, for them to operate. The designed firmware is
primarily preemption based where the DSP configures, assigns work, to
the peripherals and hardware accelerators to concurrently acquire the
signals. Upon completion of the assigned work, i.e., data
acquisition, these accelerators interrupt the DSP, which configures
DMA to write the collected data in the form of audio and vitals pages
(explained in Section \ref{filesystem}) into the SD card. This
sequence of interrupts and services happen periodically to acquire
and store the signals.

\begin{figure}
	\centering
	\includegraphics[scale = 0.5 ]{dma_interrupt}
\caption{DMA Interrupt Service Routine}
\label{dma_isr}
\end{figure}

There are two non-maskable interrupts that preempts the DSP. One is
the DMA interrupt, generated upon acquisition of audio signal of
pre-configured size and other one is the hardware interrupt generated
by the ECG front end signaling that the ECG data is ready. The
flowcharts in \FIG{dma_isr} and \FIG{ecg_isr} give the functionality
of DMA interrupt service routine (ISR) and the hardware interrupt
ISR, respectively. The ECG ISR is responsible of acquiring the
available ECG data from the Analog front end and also for acquiring
the body temperature, environment temperature, and accelerometer data.

 
 \begin{figure}
 	\centering
 	\includegraphics[scale = 1 ]{ECG_interrupt}
 	\caption{Hardware Interrupt Service Routine\label{ecg_isr}}
 \end{figure}

The following would provide insight and explain in detail the data
storage mechanism with the context of audio signal acquisition.  The
two audio channels are sampled at 12~KHz. The audio signal
acquisition and storage is the most intensive task of the entire
system and also very power hungry.  Involving DSP to do this complete
task would completely utilize the DSP's computational bandwidth while
allowing no space for DSP to do other useful work.  Also, the
overhead of doing this operation through DSP is very high.
TMS320C5515 includes 4 DMA controllers with four DMA channels each.
It can move data among internal memory, external memory, and
peripherals without intervention from the DSP and in the background
of DSP operation. So, DMA data transfer is used in audio signal
collection and storing. \FIG{DMA_Architecture} shows the DMA in the
DSP. 

 \begin{figure}
 	\centering
 	\includegraphics[scale = 0.5 ]{DMA_overview}
\caption{Conceptual Block Diagram of DMA controller. \cite{dma}}
\label{DMA_Architecture}
 \end{figure} 


The DMA is configured to collect the audio signal from the I2S
peripheral and cache it in a buffer. The I2S receive event triggers
the DMA to collect the audio signal. When the DMA fills up the
allocated buffer, it interrupts the DSP to initiate the SD card write
using another available DMA channel, and it takes care of the further
action. The system continues acquiring audio signals even while
writing data into the SD card. This is done through using DMA in
ping-pong mode. \FIG{ping_pong} shows the ping-pong operation mode.
 
 \begin{figure}
 	\centering
 	\includegraphics[scale = 0.5 ]{ping_pong}
\caption{Ping-Pong Mode for DMA Data Transfer. \cite{dma}}
\label{ping_pong}
 \end{figure}

 When the ping buffer data is being copied to the SD card , the DMA acquires the incoming data and fills it in the pong buffer. So DMA used in ping-pong mode enables the system to collect the audio signals and store the previously cached signal simultaneously. 
 
As mentioned earlier, the measurement of temperature signal is
triggered for every 500 interrupts of ECG. This adds an overhead of
checking if the interrupt counts every time the ECG interrupt is
serviced.  The measurement of temperature within the context of ECG
interrupt increases the worst-case execution time of the ISR, which
is not preferred in general for typical application. One could argue
that the system could have used a timer and generate a timer
interrupt every one second and trigger the temperature measurement
instead of tightly coupling it with the ECG ISR. The reason for not
going with this approach is the cost in terms of power for running a
dedicated timer for the temperature measurement. As long as the tasks
could be finished on available clock cycles without disturbing other
functionality of the system, it is fine even if the worst-case
execution time of the ISR increases.

 \begin{figure}
 	\centering
 	\includegraphics[scale = 0.5 ]{record_dataflow}
\caption{Flow of data in during record}
\label{record_datalow}
 \end{figure}

 \begin{figure}
	\centering
	\includegraphics[scale = 0.5 ]{play_dataflow}
\caption{Flow of data during import}
\label{play_dataflow}
\end{figure}

\subsection{Importing the Data}

Once the data has been saved to the Micro SD card, it can be read
directly by a computer. A custom file system is used to store the
data, so it cannot be directly recognized on a computer as one of the
supported file systems. A filesystem driver for the custom filesystem
is developed in Python, which could parse the data on the SD card and
create files of each of the signals. These files are read by a MATLAB
script that does post processing and generates data vectors
corresponding to each signal.  For example, in case of the body
temperature sensor, the data in the file is the raw ADC values, the
calibration details of the body temperature is embedded into MATLAB
code, so it does convert raw ADC temperature data into a meaningful
value through the calibration information. 



 \begin{figure}
	\centering  
	\includegraphics[scale = 0.4 ]{MATLAB_GUI}
\caption{MATLAB Graphic User Interface}
\label{fig:matlab_gui}
\end{figure}

MATLAB plots the data and visualizes it as shown in
\FIG{fig:matlab_gui} so that the doctors and physicians can relate
and see all the signals together. The tool also allows them to jump
to the time of interest and tune the time window shown by the tool. 


\nomenclature{RTOS}{ Real-Time Operating Systems }
\nomenclature{ISR}{ Interrupt Service Routine }

