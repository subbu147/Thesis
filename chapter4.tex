\chapter{Firmware Architecture}

\section{Overview}
The firmware for the system is implemented on Texas Instrument's TMS320C5515 DSP in C and assembly using Code Composer Studio IDE. TI provides DSP/BIOS (in later versions called SYS/BIOS)  a real time operating system for their DSPs. These operating systems provides a wide range of system services to an embedded application such as preemptive multitasking, memory management and real-time analysis. The RTOS is feature rich and helps in quicker development and testing of firmware. But the very important of drawback of using the RTOS is the trade off of performance and controllability of the DSP for quicker development. For a resource and power constrained application, it is required to squeeze out the performance from the system which requires the firmware to have complete control over the underlying hardware. Thus firmware for the system is implemented on baremetal using  C and using the chip support library (CSL) for C5515 provided by TI. 


\vfill

The functions of Software could be categorized as the following:
 \begin{description}
 	\item[$\bullet$]
Initializing System and Sensors 
 	\item[$\bullet$]
Data collection
 	\item[$\bullet$] and storage 
Power management 
 \end{description}

$//TODO$ The flow chart goes here

\section{Initializing System}
First thing the firmware does after boot-up is to initialize the clock by configuring the PLL registers. PLL is configured to generate a 100 MHz clock to tick the system. After the clocks have been set up, the multiplexed I/O pins are configured to function as a specific peripheral or a input/output pins. Then the clock and power to those I/O pins and peripherals are enabled. After the I/O pins are configured, the communication peripherals such as SPI,I2C and I2S are configured. Their mode of operation(master or slave) , frequency of operation etc are initialized and configured. DSP's I2S module is configured as the I2S master(who supply bit clock and word clock) and the Audio codec AIC3204 is configured as the I2S slave. 

\section{Initializing sensor}
Audio Codec AIC3204 is configured, through the I2C interface , to sample the audio signal at 12Khz on each of the two channels and configured to amplify the signal with automatic gain correction feature enabled. The codec is also configured to filter the audio signal using first order IIR filter and 5 BiQuad filters to improve the audio quality. 

The ECG analog front end ADS1292R is configured to sample ECG signal at 500 samples per second. ADS1292R is configured to generate a hardware interrupt to inform the DSP when the sample is ready so that DSP can read the data using SPI interface. The ECG data is again stored in a ping-pong buffer and the data is stored in the permanent storage with the same mechanism as mentioned earlier. 
The environment temperature sensor STS21 is configured to sample the temperature with a 11bit precision. This data is acquired by polling. For every 500 ECG interrupt the STS21 is polled once to collect the temperature data.  
The Analog to Digital converters on board are configured to  sample the body temperature once for every 500 interrupt of ECG.

\section{Data Collection and storage}